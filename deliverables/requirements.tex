\documentclass[12pt]{article}

\usepackage[utf8]{inputenc}
\usepackage[T1]{fontenc}
\usepackage[francais]{babel}
\usepackage[top=2cm, bottom=3cm, left=3cm, right=3cm]{geometry}
\usepackage{url}

\title{Projet PooCAv}
\author{Li XIANG, Jérôme SKODA, Joaquim LEFRANC, Hassane DIABY}
\date{2017}

\begin{document}
\fontfamily{cmr}
\maketitle
\section{Requirements}

\subsection{Propriétés fonctionnelles}
\begin{itemize}
  \item Pouvoir créer des différents types de questions:
    \begin{itemize}
      \item Choix multiple
      \item Rédaction
      \item Image
      \item Code source de divers langage
    \end{itemize}
  \item Deux types d'usagers
    \begin{itemize}
      \item L'administrateur du questionnaire (par exemple un professeur)
      \item L'utilisateur (par exemple un éleve)
    \end{itemize}
  \item Plusieurs systèmes de fonctionnement
    \begin{itemize}
      \item Evaluation
      \item Correction automatique
    \end{itemize}
	\item Système de création de cours / questionnaires
	\item Système d'inscription à un cour / questionnaire
  \item Pouvoir visualiser les résultats (score / notes)
  \item Possibilité de suivi pour les administrateur
\end{itemize}

\subsection{Propriétés non-fonctionnelles}
\begin{itemize}
  \item Déployement client/serveur: interface d'administration
	\item Le développement de l'application se fera avec le framework Play
	\item Utilisation du langage Scala
	\item Utilisation du framework Akka Http pour l'architecture client/serveur
	\item Le serveur contiendra tout les cours, questionnaires, comptes utilisateurs
\end{itemize}

\end{document}
